\section{Einleitung}
\label{sec:Einleitung}
Diese Arbeit beschäftigt sich mit der Umsetzung eines Softwareprojektes im Zuge des Modules Programmierung Projekt SS2023. Dabei soll ein selbstgewähltes Softwareentwicklungsprojekt umgesetzt und mit einer wissenschaftlichen Arbeit Dokumentiert werden. In dieser Arbeit werden alle Projektphasen der Umsetzung des Projektes beleuchtet, wobei der Fokus auf den konzeptionellen Teilen liegt. Nach der Abhandlung der einzelnen Projektphasen soll die entwickelte Software als Ergebnis der Arbeit mit ihrem Funktionsumfang und den wichtigsten Codeelementen vorgestellt werden.

\subsection{Projektumfeld}
\label{sec:Projektumfeld}
Der Praxispartner für dieses Projekt ist das Sächsische Staatsministerium für Kultus (SMK). Das SMK ist ein Ministerium der Landesregierung und fungiert als oberste Schulaufsichtsbehörde im Freistaat Sachsen. Damit ist das SMK für die Umsetzung der Bildungspolitik zuständig. Zu den Hauptaufgaben zählen vor allen Dingen die Planung und Verwaltung von Schulen und Kindertageseinrichtungen, insbesondere das Festlegen von Richtlinien und die Ausarbeitung von Lehrplänen.

Das SMK ist in vier Abteilungen mit jeweils mehreren Fachreferaten unterteilt siehe \ref{abb:Organigramm}. Zusätzlich gehört das Landesamt für Schule und Bildung (LaSuB), mit mehreren Standorten in ganz Sachsen zum nachgeordneten Bereich des SMK. Insgesamt trägt das SMK sorge für über 38.000 Beschäftigte und mehr als 685.000 Kinder und Jugendliche in sächsischen Kindertageseinrichtungen und Schulen.

Im Hauptgebäude am Carolaplatz 1 arbeiten ca. 250 Bedienstete die sich auf fünf Etagen verteilen. Das führt dazu, dass die Verfügbarkeit von Mitarbeitern für Kollegen und Referatsleiter nicht unmittelbar ersichtlich ist. Deswegen endstand in jedem Referat je nach bedarf eine eigene Anwesenheitsliste in unterschiedlichen Ausprägungen. Die Komplexität und Nutzungsfrequenz dieser Listen stieg im laufe der letzten Jahre stetig an, da die Arbeit aus dem Homeoffice und mobiles Arbeiten immer weiter in den Arbeitsalltag integriert wurden. Das führte letztlich zu einigen Problemen die mit der Entwicklung eines einheitlichen Anwesenheitsplaners für alle Referate behoben werden sollen.

Die Umsetzung des Projektes erfolgt dabei durch Referat 12 und in enger zusammenarbeit mit Referat 22 und den Datenschutzbeauftragen sowie des Informationssicherheitsbeauftragten des SMK. Als Projektleiter ist die zuständige IT-Referenten aus Referat 12 eingesetzt.

\subsection{Projektmotivation}
\label{sec:Projektmotivation}
Das Projekt zielt darauf ab, eine neue Lösung für die Anwesenheitsplanung zu finden. Derzeit erfolgt die Planung von Anwesenheiten der Bediensteten der jeweiligen Referate mithilfe von Excel-Dateien. Dieser Ansatz stößt jedoch auf verschiedene Probleme, die dringend angegangen werden müssen.

Das Hauptproblem besteht darin, dass die Funktionsweise von MS Excel in der für den Freistaat Sachsen verfügbaren Version es nicht zulässt, ein Dokument parallel durch zwei Nutzer zu öffnen oder zu bearbeiten. Dies führt zu häufigen Fehlermeldungen, wenn ein Bediensteter das Dokument bereits geöffnet hat und ein zweiter parallel darauf zugreifen möchte. Dieser Umstand führt zu frustration bei den Mitarbeitern und verursacht dadurch Fehler in der Anwesenheitsplanung. Die fehlerbehebung muss oft durch das IT-Referat durchgeführt werden und erzeugt somit zusätzlichen Arbeitsaufwand.

Ein weiterer wichtiger Aspekt ist der zunehmende Bedarf an Planung der Anwesenheiten der Mitarbeiter aufgrund der vermehrten Nutzung von Homeoffice. In der heutigen Arbeitswelt ist es für viele Unternehmen und Organisationen üblich geworden, ihren Mitarbeitern die Möglichkeit zu geben, von zu Hause aus oder mobil zu arbeiten. Dies sorgt jedoch auch für neue Herausforderungen bei der Anwesenheitsplanung. Es ist entscheidend, dass sowohl die Referatsleiter als auch die Bediensteten einen klaren Überblick über die Anwesenheit ihrer Kollegen haben, um vor Ort Tätigkeiten abzustimmen und \zB Funktionszeiten zu gewährleisten.

Angesichts dieser Probleme und Anforderungen besteht ein dringender Bedarf an einer soliden Lösung für die Anwesenheitsplanung. Eine solche Lösung sollte es den Benutzern ermöglichen, gleichzeitig auf die Anwesenheitsliste zuzugreifen und es zu bearbeiten. Darüber hinaus sollte die Lösung benutzerfreundlich sein und es den Referatsleitern und Bediensteten erleichtern, den Überblick über die Anwesenheit der Mitarbeiter zu behalten.


\subsection{Projektziel}
\label{sec:Projektziel}

Das Ziel dieses Projektes ist es, eine effiziente und Benutzerfreundliche Lösung für die Anwesenheitsplanung zu entwickeln und allen Referaten zentral bereitszustellen. Dafür soll eine geeignete Softwarelösung geschaffen und implementiert werden, welche die Einschränkungen der Excel-Datei behebt und möglichst viele Anforderungen für die Anwesenheitsplanung aus den Referaten erfüllt. Zu beachten sind hierbei die Datenschutz- und Datensicherheitsanforderung auf Grundlage der Informationssicherheitsrichtlinien des SMK.

Final soll die Planung der Anwesenheiten erleichtert und somit die Kommunikation und Zusammenarbeit zwischen den Mitarbeitern und Referatsleitern verbessert werden.

\subsection{Projektplanung}
\label{sec:Projektplanung}

Eine gründliche Projektplanung ist essenziell für ein Softwareentwicklungsprojekt um einen strukturierten und effizienten Ablauf zu gewährleisten. Für die Ablaufplanung kann auf das Grundmodell des Projektmanagements nach \cite[S.225]{dehler-2013} zurückgegriffen werden. Demnach besteht ein Projekt aus Analyse-, Entwurfs-, Implementierungs- und Testphase. Für die Entwicklung wird das Wasserfallmodell als standardisiertes Softwareentwicklungsverfahren verwendet, da das umzusetzende Projekt nur über einen kleinen zeitlichen Rahmen verfügt und überschaubar Komplex ist.

%TODO: Wasserfallmodell als Quelle aus Buch!
Für eine bessere Übersicht des Projektablaufs wurde ein Gantt Diagramm für die Zeitplanung mit Meilensteinen erstellt, siehe Anhang \ref{abb:Gantt}. Für die Umsetzung des Projektes nimmt die Analysephase mit der IST- und SOLL-Analyse eine besonders wichtige Positionen ein. In der IST-Analyse werden die bestehenden Anwesenheitsplanungsmethoden der Referate und deren Schwachstellen Untersucht werden. Der SOLL-Zustand wird durch eine gründliche Anforderungsanalyse ermittelt, um die Anforderungen und Ziele des Projekts klar zu definieren. Hierbei sollen sowohl von Mitarbeitern gewonnene Informationen, als auch durch die Analyse der vorhandenen Datenbestände in den Anwesenheitslisten ausgewertet werden um relevante Anforderungen zu identifizieren. Diese Anforderungen dienen als Grundlage für die Design- und Implementierungsphase, in der die eigentliche Entwicklung der Software stattfindet.

Nach Abschluss der Implementierung soll ein lauffähiger Prototyp entstehen, der dann für Tests verwendet werden kann. Sobald die Testphase erfolgreich abgeschlossen ist kann die Software in die Produktivumgebung migriert werden. Die Migration bildet den Abschluss des Projektes.


