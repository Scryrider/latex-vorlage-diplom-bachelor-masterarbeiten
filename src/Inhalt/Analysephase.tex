\section{Analysephase}
\label{sec:Analysephase}
%TODO:Quelle Recherche Problemdefinition / Requiremants Engeneering
Die Analysephase eines Softwareentwicklungsprojekts spielt eine entscheidende Rolle für den Erfolg eines Projekts, da hier die Anforderungen an das zu entwickelnde System erhoben werden. Zur Ermittlung der Anforderungen soll eine Systemanalyse und eine Anforderungsanalyse durchgeführt werden, die Einblicke in die bisherigen Prozesse der Anwesenheitsplanung geben sollen. Für die Informationsbeschaffung soll auf die Analysemethode der Datenanalyse gesetzt werden. Um die so gewonnenen Daten zu Validieren und auch subjektive Verbesserungsvorschläge zu berücksichtigen werden ebenfalls Intervierws mit den Mitarbeitern durchgeführt. Die dabei erhobenen Daten werden dann Aufbereitet und Modelliert.

In den folgenden Abschnitte beschreiben die Vorgehensweise und Ergebnisse der durchgeführten Analysen. Die dabei gewonnenen Erkenntnisse bilden die grundlage für die weiteren Entscheidungen für die Umsetzung des Projektes.


\subsection{Analyse des Ist-Zustandes}
\label{sec:Ist-Zustand}

%TODO:Bestehende Anwesenheitsplanungsmethoden (Intervierws mit Mitarbeitern + Analyse der Exeldateien)
%TODO:Schwachstellen und Herausforderungen im aktuellen Prozess (Recht,keine Intigrität,kein mehrfacher zugriff,nur ecxel zugreifbar)

% Bestehende Anwesenheitsplanungsmethoden (Datenstruktur/Tabellenaufbau)
% Schwachstellen und Herausforderungen im aktuellen Prozess
Um eine verbesserte Variante eines Anwesenheitsplaners zu erstellen ist es notwendig die bestehenden Systeme und Abläufe zu verstehen. Dafür wurden Interviews mit Mitarbeitern aus verschiedenen Referaten geführt und analog dazu die verschiedenen Excel-Dateien analysiert. Dabei lag der Fokus auf der Analyse des Prozesses der Anwesenheitsplanung, um diesen durch die neue Variante des Anwesenheitsplaners bestmöglich zu unterstützen. 

Die Interviews ergaben, dass das Excel-Dokument meist in einem Netzlaufwerk aufbewahrt wird um für die Rederatsmitglieder zugänglich zu sein. Das Dokument wird nicht anders behandelt als andere Referatsdaten was bedeutet, dass es keine spezielle Rechteverwaltung gibt. Das führt zu Fehler und Inkonsistenzen durch versehentliches ändern anderer Datensätze, da schreibrechte auf das ganze Dokument vergeben sind.

Als größte Problem wurde die Tatsache, dass nur ein Nutzer gleichzeitig auf das Excel-Dokument zugreifen kann festgestellt. Dies führt dazu, dass für eine unbestimmte Zeit keine Möglichkeit besteht den eigenen Status zu ändern. Meistens tritt der Umstand auf, wenn ein Mitarbeiter die Datei im Hintergrund geöffnet hat diese aber nicht nutzt.   

Für die Datenanalyse wurden von den beteiligten Referaten eine Kopie ihrer Excel Datei zur Verfügung gestellt. Hauptsächlich sollte der grafische Aufbau der Tabelle und die enthaltenen Informationen der Tabellen untersucht werden. Die Analyse ergab, dass alle Anwesenheitsplaner Tabellen ähnlich aufgebaut sind. Unterschiede ergaben sich meist nur in der Bezeichnung der Anwesenheitszustände und die Granularität der Zeitabschnitte. Um eine Referenz für die weitere Entwicklung zu haben wurde aus den Referats-Tabellen eine Modelltabelle für die Anwesenheitsplanung erstellt.
%Basierend auf diesen Erkenntnissen wird deutlich, dass eine dringende Notwendigkeit besteht, den Anwesenheitsplaner zu verbessern. Eine effektive Lösung sollte die Möglichkeit bieten, dass mehrere Nutzer gleichzeitig auf die Anwesenheitsdaten zugreifen und diese aktualisieren können. Die Automatisierung von Prozessen und die Integration einer Datenbank könnten die Genauigkeit, Effizienz und Benutzerfreundlichkeit des Anwesenheitsplaners erheblich verbessern.

\subsection{Anforderungen an den Anwesenheitsplaner}
\label{sec:Soll-Zustand}
%TODO:UseCase
%TODO:Kernfunktionen
%TODO:Umsetzungsrichtlinien
%TODO:Berechtigungensonzept

\subsection{ Datenschutz und Datensicherheitsanalyse}
\label{sec:Datenschutz}
%TODO:Personenbezogene Daten?
%TODO:Anforderungen Datenschutzkonzept
%TODO:Datensicherheit / Risikoanalyse