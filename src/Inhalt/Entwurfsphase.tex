\section{Entwurfsphase}
\label{sec:Entwurfsphase}
In der Entwurfsphase sollen aus den Anforderungen der Analysephase Modelle der zu entwickelten Software entstehen, die die konkreten Hardware- und Softwarebezogenen Anforderungen berücksichtigen. Diese Modelle gelten dann als unmittelbare Vorlage für die sich anschließende Implementationsphase. (vgl. \cite[S. 69]{dumke-2003})

Zu Beginn soll eine Marktrecherche durchgeführt werden, um bestehende Lösungen zu analysieren und auf Eignung zu Prüfen. Mit den gewonnen Erkenntnissen soll dann abgewägt werden ob Standartsoftware beschafft wird oder eine eigenentwicklung Veranlasst wird. Das erfolgt im Rahmen einer Variantendiskussion, in der verschiedene Ansätze und Technologien zur Umsetzung der Software bewertet werden.

Letztendlich soll der Programm- und Datenentwurf mithilfe geeigneter Modelle erarbeitet werden. Dies umfasst die Festlegung der Softwarearchitektur, der Datenstruktur sowie etwaiger Schnittstellen und der grafische Oberfläche.

\subsection{Marktrecherche}
\label{sec:Marktrecherche}
%TODO:Schicke Tabelle aus Excel Bauen
Im Rahmen der Marktanalyse wurden drei verschiedene Softwarelösungen für die Anwesenheitsplanung verglichen. Das Ziel dieser Analyse war es, Akquisisationsoptionen für das geplante Softwareprojekt zu ermitteln. Um Entscheidungen über die Eignung treffen zu können, wurden die Softwareprodukte auf den Erfüllungsgrad der Anforderungen geprüft. Dafür wurde eine Bewertungsmatrix erstellt, die funktionale- als auch nichtfunktionale Anforderungen beinhaltete. Dabei wurden die Kriterien so gewählt und gewichtet das zwingend erforderliche Anforderungen höher gewichtet wurden als optionale Anforderungen.

%TODO:Die Schicke Tabelle hier einfügen!!!

Die erste betrachtete Software, Programm A, zeichnete sich durch eine umfangreiche Funktionalität aus, die sowohl für kleine als auch große Projekte geeignet war. Die Benutzerfreundlichkeit wurde jedoch als etwas komplex und steil eingestuft, was eine gewisse Einarbeitungszeit erforderte. Zudem war der Support nur eingeschränkt verfügbar und die Preisgestaltung vergleichsweise hoch.

Das zweite Programm, Programm B, überzeugte hingegen durch eine intuitive Benutzeroberfläche und eine gute Skalierbarkeit. Es bot eine solide Funktionalität für die meisten Anforderungen des geplanten Projekts. Der Support war zuverlässig und die Preisgestaltung angemessen. Jedoch fehlten einige spezifische Funktionen, die für das Projekt von hoher Bedeutung waren.

Das dritte Programm, Programm C, präsentierte sich als eine kostengünstige Option mit einer breiten Palette an Funktionen und einem guten Support. Es war jedoch in Bezug auf die Skalierbarkeit begrenzt und hätte möglicherweise Probleme mit wachsenden Projektanforderungen gehabt.

Nach eingehender Analyse der drei Programme wurde festgestellt, dass Programm B die beste Wahl für das Softwareentwicklungsprojekt darstellt. Es vereinte eine benutzerfreundliche Oberfläche, eine solide Funktionalität und eine angemessene Preisgestaltung. Die fehlenden spezifischen Funktionen konnten durch Anpassungen und Erweiterungen ausgeglichen werden, während der gute Support das reibungslose Funktionieren des Projekts gewährleisten würde. Durch die Auswahl von Programm B ist das Projektteam zuversichtlich, eine effiziente und erfolgreiche Softwareentwicklung zu ermöglichen.


\subsection{Variantendiskussion}
\label{sec:Marktrecherche}

\subsection{Programm und Datenentwurf}
\label{sec:ProgrammUDatenentwurf}