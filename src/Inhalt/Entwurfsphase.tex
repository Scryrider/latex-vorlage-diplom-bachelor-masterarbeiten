\section{Entwurfsphase}
\label{sec:Entwurfsphase}
In der Entwurfsphase sollen aus den Anforderungen der Analysephase Modelle der zu entwickelten Software entstehen, die die konkreten Hardware- und Softwarebezogenen Anforderungen berücksichtigen. Diese Modelle gelten dann als unmittelbare Vorlage für die sich anschließende Implementationsphase. (vgl. \cite[S. 69]{dumke-2003})

Zu Beginn soll eine Marktrecherche durchgeführt werden, um bestehende Lösungen zu analysieren und auf Eignung zu Prüfen. Mit den gewonnen Erkenntnissen soll dann abgewägt werden ob Standartsoftware beschafft wird oder eine eigenentwicklung Veranlasst wird. Das erfolgt im Rahmen einer Variantendiskussion, in der verschiedene Ansätze und Technologien zur Umsetzung der Software bewertet werden.

Letztendlich soll der Programm- und Datenentwurf mithilfe geeigneter Modelle erarbeitet werden. Dies umfasst die Festlegung der Softwarearchitektur, der Datenstruktur sowie etwaiger Schnittstellen und der grafische Oberfläche.

\subsection{Marktrecherche}
\label{sec:Marktrecherche}

\subsection{Variantendiskussion}
\label{sec:Marktrecherche}

\subsection{Programm und Datenentwurf}
\label{sec:ProgrammUDatenentwurf}